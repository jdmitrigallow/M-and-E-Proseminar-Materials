%%%%BASIC PREAMBLE
\input{/Users/jdg83/Dropbox/0000Desktop/Preambles/preamble1}
%%%%ADDITIONAL PACKAGES
\usepackage{marvosym}
\usepackage{fullpage}
\usepackage{layout}
\hoffset = -20pt
\usepackage{longtable}
\pagestyle{empty}
\usepackage{colortbl}
\usepackage{tabularx}
%%%%ADDITIONAL NEW COMMANDS
\newcommand{\grey}{ \rowcolor[gray]{.85}}
\newcommand{\white}{\rowcolor[gray]{1}}
\renewcommand{\qe}{\begin{enumerate}[topsep=5pt,itemsep=0ex,partopsep=0ex,parsep=0ex]}
\renewcommand{\ze}{\vspace{-10pt}	\end{enumerate} 	\\}
\newcommand{\zef}{\end{enumerate}}
%\newcommand{}{}
%%%%NEW THEOREMS
%\newtheorem{theorem}{Theorem}
%\newtheorem{lemma}{Lemma}
%%%%DOCUMENT
\usepackage[nottoc, notlof, notlot]{tocbibind}
\begin{document}
\noindent\begin{longtable}{l l}
%%%%%COURSE
{\s{course}}	& \s{Phil 2400: Metaphysics \emph{\&} Epistemology Core Seminar}	\\
	&	\\
%%%%%INTSTRUCTOR
{\s{instructor}}	&   \vspace{-12pt}	\\
& \begin{minipage}{390pt}
J.~Dmitri Gallow		(\Letter:	 \href{mailto:jdmitrigallow@pitt.edu}{jdmitrigallow@pitt.edu})		
\end{minipage} \\
\\
%%%%%MEETING TIMES
{\s{seminar times}}	&   \vspace{-12pt}	\\
& \begin{minipage}{390pt}
	Tuesdays, 19:00--21:30	\\
	Room 1008A, Cathedral of Learning
\end{minipage} \\
			\\
%%%%%OFFICE HOURS
{\s{office hours}}	&Mondays and Wednesdays, 16:00--17:00	\\
					&Room 1029-D, Cathedral of Learning
					\vspace{12pt}	\\
					
%%%%%DESCRIPTION
{\s{course goals}}	&   \vspace{-12pt}	\\
& \begin{minipage}{390pt}
The primary goals of this course are three-fold: firstly, to introduce first-year graduate students to a few central debates and theories within contemporary philosophy of language, logic, metaphysics, and epistemology.  Secondly, to acquaint them with some of the skills they will need to fruitfully engage with positions and debates in the philosophy of language, logic, metaphysics, and epistemology.   And thirdly, to further develop their ability to write, present, and fruitfully engage with philosophical arguments and positions, especially those in philosophy of language, logic, metaphysics, and epistemology.
\end{minipage} \\
		&	\\
%%%%%TEXTS
%	{\s{course text}}	&   \vspace{-12pt}	\\
%	& \begin{minipage}{390pt}
%	\nocite{hausman:2006}
		%%%%BIBLIOGRAPHY
%\renewcommand{\section}[2]{}%
%\bibliographystyle{/Users/jdg83/Desktop/General_Bib/test2}
%\bibliography{/Users/jdg83/Desktop/General_Bib/general}
%	\end{minipage} \\
%		&	\\
%%%%EVALUATION
{\s{evaulation}}	&   \vspace{-12pt}	\\
& \begin{minipage}{390pt}
Final grades will be determined by 4 components:
\begin{center}
\begin{tabular}{l l}
Responses	&			---			\\
Papers			&			60\% (20\% each)  \\
Presentation	&		20\%	\\
Participation	&		20\%	
\end{tabular}
\end{center}

\textsc{Responses}: Every week, starting with our second meeting, e-mail me a short paragraph response or question by Monday evening (7pm) about one of the readings for the subsequent class. These are informal and meant to give me a sense for what is troubling people in the readings. They won't be graded, but must be completed for course credit.\\

\textsc{Papers}: There will be three papers of between 5 and 7 full pages (double spaced, 12pt.). Respect these length limits. Assignment distribution dates are below. You should have about two to three weeks to complete each paper. Late assignments are docked a half letter grade per day, barring special excuses. \\

\textsc{Presentation}: 	You'll sign up to present material from one of the later readings and help lead a subsequent discussion on the topic. More instructions are on the \e{presentations} handout.\\

\textsc{Participation}: It is important that you come to seminar prepared to actively (and respectfully)  participate in the discussion. This means 1) that you should have done all the required readings;  2) that you should contribute, without dominating, the discussion; and 3) that you should be respectful of your fellow classmates. You should take a look at Chalmer’s \href{http://consc.net/norms.html}{guidelines} for respectful, constructive, and inclusive philosophical discussion to get a more concrete idea of what I mean by treating your classmates respectfully.\\

The seminar is intended to be an \e{active} learning environment.   I (normatively) expect students to arrive with questions and concerns, and to be actively engaged throughout the seminar.

%%%BREAK
\end{minipage} \\
\pagebreak
%%%%EVALUATION CONT
%{\s{evaluation}}	&		\\
%{\s{(con't)}}	&   \vspace{-24pt}	\\
%& \begin{minipage}{390pt}
%\textsc{Participation}: It is important that you come to seminar prepared to actively and respectfully  participate in the discussion. This means 1) that you should have done all the required readings;  2) that you should contribute to the discussion; and 3) that you should be respectful of your fellow classmates. You should take a look at Chalmer’s \href{http://consc.net/norms.html}{guidelines} for respectful, constructive, and inclusive philosophical discussion to get a more concrete idea of what I mean by treating your classmates respectfully.\\
%
%The seminar is intended to be an \e{active} learning environment.  I am here to exposit the assigned reading and provide relevant background information, but I (normatively) expect those attending the seminar to arrive with questions and concerns, and be actively engaged throughout the seminar.

%%%BREAK
%\end{minipage} \\
%\\
%%%SCHEDULE
{\s{schedule}}	&   \vspace{-12pt}	\\
& \begin{minipage}{390pt}
\begin{tabularx}{\textwidth}{X} 
{~}\hfill \textbf{Philosophy of Language}		\hfill{~}	\\
%
\grey
 	August 28th: Frege's Puzzle
		\qe
		\p[] Syllabus 
		\p[] Frege, \emph{On Sense and Reference}
		\ze
%
\white
	September 4th: Introduction to Modal Logic
		\qe
		\p[] Garson, \e{Modal Logic}, \S\S 1, 2, 6--8
		\p[] Sider, \e{Logic for Philosophy}, pp.~171--187
		\ze 
%
\grey
	September 11th: Names, Descriptions, Necessity, and A Priority
		\qe
		\p[] Kripke, \emph{Naming \& Necessity}, Lectures I and II
		\ze 
%
	\white
		September 18th: The \e{De Se} 
		\qe
		\p[] Lewis, \e{Attitudes De Dicto and De Se}
		\p[] Perry, \e{The Problem of the Essential Indexical}  
		\p[] \tbf{First Paper Assigned}
		\p[] \tbf{First Presentation}
		\ze 
%
%	\grey
%		September 25th: no class
%		\qe
%		\p[] 
%		\ze 
%
	\grey
		September 25th: Gricean Implicature (alternatively, catch up, if we need additional time)
		\qe
		\p[] Grice, \emph{Logic and Conversation}
		\p[] \tbf{First Paper Due}
		\ze 
%
\white{~}\hfill \textbf{Metaphysics} \hfill{~}		\\
	\white
		October 2nd: Vagueness
		\qe
		\p[] Keefe, \e{Supervaluationism}
		\p[] Fine, \e{Vagueness, Truth, and Logic} (\e{optional})
		\p[] Williamson, \e{Vagueness and Ignorance}
		\p[] Williamson, \e{Vagueness, chapter 5} (\e{optional})
		\p[] \tbf{Second Presentation}
		\ze 
%
	\grey
		October 9th:  Persistence
		\qe
		\p[] Kurtz, \e{Introduction to Persistence: What's the Problem?}
		\p[] Sider, \e{Four-Dimensionalism}
		\p[] \tbf{Third Presentation}
		\p[] \tbf{Second Paper Assigned}
		\ze 
%
	\white
		October 16th:  no class (Fall break)   
		\qe
		\p[] 
		\ze  
%
	\grey
		October 23rd:  Material Constitution
		\qe
		\p[] Wasserman, \emph{Material Constitution}
		\p[] Lewis, \e{Counterparts of Persons and their Bodies} 
		\p[] Lewis, \e{Counterpart theory and Quantified Modal Logic} (\e{optional})
		\p[] \tbf{Fourth Presentation}
		\ze  
%
	\white
		October 30th: Conditionals (alternatively, catch up, if we need additional time)
		\qe
		\p[] Stalnaker, \e{A Theory of Conditionals}
		\p[] Lewis, \e{Counterfactuals}, chapter 1
		\p[] \tbf{Second Paper Due}
		\ze 
%
\end{tabularx}
\end{minipage}\\
\newpage
%%%SCHEDULE
{\s{schedule}}	&  \\
{\s{(con't)}}		&		\vspace{-24pt}	\\
& \begin{minipage}{390pt}
\begin{tabularx}{\textwidth}{X} 
\white
%
{~}	 \hfill \textbf{Epistemology}		\hfill {~}	\\
	\grey
		November 6th: Skepticism and Contextualism
		\qe
		\p[] Lewis, \e{Elusive Knowledge}
		\p[] DeRose, \e{Solving the Skeptical Problem} (\e{optional})
		\p[] \tbf{Fifth Presentation}
		\ze 
%
	\white
		November 13th: Skepticism and Externalism
		\qe
		\p[] Williamson, \emph{Knowledge and its Limits}, selections
		\p[] \tbf{Third Paper Assigned}
		\p[] \tbf{Sixth Presentation}
		\ze 
%
	\grey
		November 20th: Partial Belief
		\qe
		\p[] Foley, \e{The Epistemology of Belief and the Epistemology of Degrees of Belief}
		\p[] Strevens, \e{Notes on Bayesian Confirmation Theory}, \S\S 1--4
		\p[] \tbf{Seventh Presentation}
		\ze 
%
	\white
		November 27th: Partial Belief and the \e{De Se}
		\qe
		\p[] Elga, \e{Self-Locating Belief and the Sleeping Beauty Problem}
		\p[] Titelbaum, \e{Ten Reasons to Care about the Sleeping Beauty Problem}
		\p[] Arntzenius, \e{Some Problems for Conditionalization and Reflection} (\e{optional})
		\ze 
%
%
	\grey
		December 4th: Peer Disagreement (alternatively, catch up, if we need additional time)
		\qe
		\p[] Elga, \e{Reflection and Disagreement}
		\p[] Weatherson, \e{Disagreements, Philosophical and Otherwise}
		\p[] \tbf{Third Paper Due}
		\ze 
%
\end{tabularx}
\end{minipage}\\
\\
%%%
\s{academic}		&	\\
\s{integrity}		& \vspace{-24pt}		\\
&	\begin{minipage}{390pt}
	Students in this course will be expected to comply with the University of Pittsburgh's Policy on Academic Integrity. Any student suspected of violating this obligation for any reason during the semester will be required to participate in the procedural process, initiated at the instructor level, as outlined in the University Guidelines on Academic Integrity. This may include, but is not limited to, the confiscation of the examination of any individual suspected of violating University Policy. Furthermore, no student may bring any unauthorized materials to an exam, including dictionaries and programmable calculators.
	\end{minipage} \\
	\\
	%%%DISABILITY STATEMENT
\s{disability}		&	\\
\s{services}		& \vspace{-24pt}		\\
&	\begin{minipage}{390pt}
If you have a disability for which you are or may be requesting an accommodation, be sure to contact me within the first two weeks of the semester, as well as Disability Resources and Services (DRS), 140 William Pitt Union, (412) 648-7890, \url{drsrecep@pitt.edu}, (412) 228-5347 for P3 ASL users. DRS will verify your disability and determine reasonable accommodations for this course.
	\end{minipage} \\
	\\
	%%%LAPTOP POLICY
\s{laptop}		&	\\
\s{policy}		& \vspace{-24pt}		\\
&	\begin{minipage}{390pt}
As a general rule,  laptops and smart phones are not permitted during class.  If you have some good reason for requiring a laptop or a cell phone during class, come speak to me about it in office hours.
	\end{minipage} \\
	\\
	\newpage
	%%%LAPTOP POLICY
\s{recording}		&	\\
\s{policy}		& \vspace{-24pt}		\\
&	\begin{minipage}{390pt}
To ensure the free and open discussion of ideas, students may not record classroom lectures, discussion and/or activities without the advance written permission of the instructor, and any such recording properly approved in advance can be used solely for the student's own private use.
	\end{minipage} \\
	\\
	%%%Revision Policy
\s{schedule}		&	\\
\s{revision}		& \vspace{-24pt}		\\
&	\begin{minipage}{390pt}
As the course progresses, the course schedule may be revised.  If it is, I will notify all enrolled students via email and post an updated syllabus to Courseweb.
	\end{minipage} 
\end{longtable}

%%%%BIBLIOGRAPHY
%\bibliographystyle{/Users/jrg349/Desktop/General_Bib/test2}
%\bibliography{/Users/jrg349/Desktop/General_Bib/general}

\end{document}