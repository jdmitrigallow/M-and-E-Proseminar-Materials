\documentclass[landscape, two column, full page,reqno]{article}
\usepackage{mathrsfs}
\usepackage{amsmath,amssymb,amsthm}
\usepackage[adobe-garamond]{mathdesign}
\AtBeginDocument{%
  \let\mathbb\relax
  \DeclareMathAlphabet\PazoBB{U}{fplmbb}{m}{n}%
  \newcommand{\mathbb}{\PazoBB}%
  \let\mathcal\relax
  \DeclareMathAlphabet{\OMScal}{OMS}{cmsy}{m}{n}
  \newcommand{\mathcal}{\OMScal}%
}
\usepackage{enumitem}
\usepackage{fontspec}
\usepackage{tikz}
\usetikzlibrary{arrows}
\usepackage{multicol}
\setmainfont[Numbers={Proportional,OldStyle}]{Adobe Garamond Pro}
%NATBIB
\usepackage[comma]{natbib}
%HYPERREF PACKAGE
\usepackage{xcolor}
\PassOptionsToPackage{hyphens}{url}
\usepackage[backref=page,linktocpage=true,colorlinks]{hyperref}
\renewcommand{\backrefxxx}[3]{[\hyperlink{page.#1}{#1}]}
\hypersetup{
    colorlinks = true,
    citecolor = blue,
    urlcolor = blue,
    filecolor = blue,
    linkcolor = blue,
}
%PATCH TO ONLY HYPERLINK YEAR OF CITATION
\usepackage{etoolbox}
\makeatletter
% Patch case where name and year are separated by aysep
\patchcmd{\NAT@citex}
  {\@citea\NAT@hyper@{%
     \NAT@nmfmt{\NAT@nm}%
     \hyper@natlinkbreak{\NAT@aysep\NAT@spacechar}{\@citeb\@extra@b@citeb}%
     \NAT@date}}
  {\@citea\NAT@nmfmt{\NAT@nm}%
   \NAT@aysep\NAT@spacechar\NAT@hyper@{\NAT@date}}{}{}
% Patch case where name and year are separated by opening bracket
\patchcmd{\NAT@citex}
  {\@citea\NAT@hyper@{%
     \NAT@nmfmt{\NAT@nm}%
     \hyper@natlinkbreak{\NAT@spacechar\NAT@@open\if*#1*\else#1\NAT@spacechar\fi}%
       {\@citeb\@extra@b@citeb}%
     \NAT@date}}
  {\@citea\NAT@nmfmt{\NAT@nm}%
   \NAT@spacechar\NAT@@open\if*#1*\else#1\NAT@spacechar\fi\NAT@hyper@{\NAT@date}}
  {}{}
\makeatother
%
%Titlesec package
\usepackage{titlesec}
%Centering and readjusting size of headings
\titleformat{\section}[hang]
{\normalfont\sc\filcenter}{\thesection}{1em}{}
\titleformat{\subsection}[hang]
{\normalfont\sc\filcenter}{\thesubsection}{1em}{}
\titleformat{\subsubsection}[hang]
{\normalfont\sc\filcenter}{\thesubsubsection}{1em}{}
			% in the document preamble: 
				\let\endgraf\par % because LaTeX doesn't like \par 
			% in some command arguments 
				\let\subtitlefont\normalfont % or whatever 
				
%FOOTNOTE SPACING
\usepackage[hang,multiple,splitrule]{footmisc}
\setlength{\footnotemargin}{4mm}

\newcommand{\qd}{\begin{quote}\begin{description}  [align=left,style=nextline,leftmargin=*,labelsep=0pt,font=\normalfont]}
\newcommand{\zd}{\end{description}\end{quote}}
\newcommand{\qe}{\begin{enumerate}[align=left,style=nextline,leftmargin=17pt,labelsep=5pt,font=\normalfont]}
\newcommand{\qei}{\begin{enumerate}[align=left,style=nextline,leftmargin=15pt, labelsep=10pt,font=\normalfont]}
\newcommand{\ze}{\end{enumerate}}
\newcommand{\p}{\item}
\newcommand{\e}{\emph}
\newcommand{\s}{\textsc}
\newcommand{\tbf}{\textbf}
\newcommand{\fn}{\footnote}
\newcommand{\argu}[2]{\begin{center}\begin{minipage}{#1} \begin{enumerate}
	#2
\end{enumerate}
\end{minipage}  
\end{center}}
%GRAPHICX PACKAGE
\usepackage{graphicx}
\graphicspath{{/Users/jdg83/Desktop/Figures/}}
\usepackage{xcolor}
\usepackage{fancybox}

\definecolor{ShadowColor}{RGB}{30,150,190}

\makeatletter
\newcommand\Cshadowbox{\VerbBox\@Cshadowbox}
\def\@Cshadowbox#1{%
  \setbox\@fancybox\hbox{\fbox{#1}}%
  \leavevmode\vbox{%
    \offinterlineskip
    \dimen@=\shadowsize
    \advance\dimen@ .5\fboxrule
    \hbox{\copy\@fancybox\kern.5\fboxrule\lower\shadowsize\hbox{%
      \color{gray}\vrule \@height\ht\@fancybox \@depth\dp\@fancybox \@width\dimen@}}%
    \vskip\dimexpr-\dimen@+0.5\fboxrule\relax
    \moveright\shadowsize\vbox{%
      \color{gray}\hrule \@width\wd\@fancybox \@height\dimen@}}}
\makeatother

\newcommand{\csbox}[2]{\begin{center}
\Cshadowbox{
\begin{minipage}{#1}
	#2
\end{minipage}}
\end{center}
}


\title{Frege, \e{On Sense and Reference}}
\date{August 28th, 2018}
\author{M\e{{\fontspec{Minion Pro} \&}}E Core}

\usepackage{layout}
\voffset = -40pt
\textheight = 450pt
\setlength{\columnsep}{20pt}
\begin{document}
%\layout
\twocolumn[{%
 \centering
\maketitle
}]

\section{Frege's First Puzzle}
\qe
\p `Hesperus' is the name the Greeks gave to the first star visible in the evening sky (\e{the evening star}).  `Phosphorus' is the name they gave to the last star visible in the morning sky (\e{the morning star}).  It turns out that neither Hesperus nor Phosphorus is a star.  Rather, they are both the planet Venus.
\p Consider the following two claims:
	\qe
	\p[(\tbf{A})] Hesperus $=$ Hesperus
	\p[(\tbf{B})] Hesperus $=$ Phosphorus 
	\ze 
	(\tbf{A}) is \e{a priori}.  (\tbf{B}) is not.  The discovery that (\tbf{B}) was true was a genuine advance in our knowledge.  Not so for (\tbf{A}).  So (\tbf{A}) and (\tbf{B}) must convey different information.  So they must say different things.
\p  Suppose that identity is a relation between objects.  Then:  
\qe \p `Hesperus' refers to Venus. So (\tbf{A}) must say that Venus (the object referred to by `Hesperus') bears the identity relation to Venus (the object referred to by `Hesperus').  
	\p `Phosphorus' refers to Venus.  So (\tbf{B}) must say that Venus (the object referred to be `Hesperus') bears the identity relation to Venus ( the object referred to by `Phosphorus').
	\ze 
So (\tbf{A}) and (\tbf{B}) must say the same thing.
\p So we've seen a compelling argument that (\tbf{A}) and (\tbf{B}) say the same thing; and another compelling argument that (\tbf{A}) and (\tbf{B}) say different things.  This, roughly, is Frege's first puzzle.
\p In his \e{Begriffschrift}, Frege tried to solve the puzzle by rejecting the claim that identity, $=$, is a relation between objects.  Rather, he contended, it is a relation between \e{names}.
	\qe
	\p Thus, `Hesperus $=$ Phosphorus' doesn't say that Venus bears the identity relation to Venus.  Rather, it says that the names `Hesperus' and `Phosphorus' co-refer.
	\p So (\tbf{A}) asserts a different relation than (\tbf{B}), and the puzzle is solved.
	\ze 
\p However, Frege came to reject this solution.  Here are two arguments against it:\label{pitfall}
	\qe
	\p The claim (\tbf{B}) appears to be {about} the world, and not about our way of talking about it.  It seems that (\tbf{C}) says the same thing as (\tbf{B}) does,
		\qe
		\p[(\tbf{C})] $E \sigma \pi \epsilon \rho o \varsigma = \Phi \omega \sigma \varphi$\'{o}$\rho o \varsigma$
		\ze 
	Since, after all, (\tbf{C}) is just (\tbf{B}) translated into Greek.  However, the claim that `$E \sigma \pi \epsilon \rho o \varsigma$' and `$\Phi \omega \sigma \varphi$\'{o}$\rho o \varsigma$' co-refer is a completely different claim than the claim that `Hesperus' and `Phosphorus' co-refer.  So if we accept Frege's first solution, then we're forced to say that (\tbf{B}) and (\tbf{C}) say different things.
	\p Frege's first solution doesn't solve the problem in general.  Consider (\tbf{D}) and (\tbf{E}):
		\qe
		\p[(\tbf{D})] The inventor of bifocals invented bifocals.
		\p[(\tbf{E})] Benjamin Franklin invented bifocals.
		\ze 
	(\tbf{D}) is \e{a priori}.  (\tbf{E}) is not.  Learning (\tbf{E}) is a genuine advance in knowledge.  Not so for (\tbf{D}).  So (\tbf{D}) and (\tbf{E}) must say different things.
	\ze 
\p One of the central goals of Frege's \e{On Sense and Reference} is to advance a new solution to the puzzle.  

\subsection{Compositionality, and the Puzzle again}
\p Frege accepted a principle of the \e{Compositionality of Meaning}, which we can gloss roughly like this:
	\qd
	\p[\s{Compositionality of Meaning}] The meanings of whole sentences are determined by the meanings of the parts which compose them.
	\zd 
	\qe
	\p Thus, e.g., the meaning of `Daniel chews' should be determined by the meaning of `Daniel' and the meaning of `chews', together with some simple rules stating how to get whole meanings out of these constituent meanings.
	\p The important point for our purposes is this: the principle of compositionality tells us that exchanging two parts with the same meaning shouldn't make a difference to the meaning of the whole.  So, e.g., if `masticate' has {the same meaning} as `chew', then `Daniel chews' will have the same meaning as `Daniel masticates'.
	\ze 
\p With this behind us, we can re-present the first puzzle like so: the following four claims are individually plausible but jointly inconsistent
	\qe
	\p \s{Compositionality of Meaning} \label{1}
	\p The meaning of `Hesperus' is its referent---the planet Venus.\label{2}
	\p The meaning of `Phosphorus' is its referent---the planet Venus.\label{3}
	\p (\tbf{A}) and (\tbf{B}) have different meanings.\label{5}
	\ze  
\subsection{Frege's Solution: Sense and Reference}
\p Frege's new solution to the puzzle finds a way of rejecting \eqref{2} and \eqref{3} while avoiding the pitfalls we saw above (\e{i.e.}, \ref{pitfall}).  His solution is to postulate two distinct aspects of meaning, which he labels \e{sense} (or \e{sinn}) and \e{reference} (or \e{bedeutung}).
	\qe
	\p The \e{referent} (\e{bedeutung}) of an expression is the object it designates.
	\p The \e{sense} (\e{sinn}) of an expression is the `mode of presentation' with which the referent is designated.
	\ze 
\p Even though `Hesperus' and `Phosphorus' have the same referent, they have different senses.  For `Hesperus', the mode of presentation is `the first star visible in the evening sky.  For `Phosphorus', the mode of presentation is `the last star visible in the morning sky'.
	\qe
	\p Thus the puzzle is solved: (\tbf{A}) and (\tbf{B}) have different meanings, since the sense of `Hesperus' differs from the sense of `Phosphorus'.
	\p Moreover,  we don't run into the same problems we encountered above.  Frege believes that expressions in different languages can have the same sense.  So `$E \sigma \pi \epsilon \rho o \varsigma$' and `Hesperus' may have the same sense.  And similarly  `$\Phi \omega \sigma \varphi$\'{o}$\rho o \varsigma$' and `Phosphorus' may have the same sense.  Thus `Hesperus $=$ Phosphorus' may have the same meaning as `$E \sigma \pi \epsilon \rho o \varsigma = \Phi \omega \sigma \varphi$\'{o}$\rho o \varsigma$'.
	\p Similarly, because the solution did not depend upon the meaning of `$=$', we may say that, even though the \e{referent} of `The inventor of bifocals' is the same  as the \e{referent} of `Benjamin Franklin', they nevertheless differ in meaning, because they differ in  \e{sense}.  So `The inventor of bifocals invented bifocals' has a different meaning than `Benjamin Franklin invented bifocals'.
	\ze 
\p What are senses?  
	\qe
	\p Frege is clear that they are \e{not} ideas.  As Frege is using the term `idea', an idea is something like a mental image.  It is important for him that ideas are not any part of meaning, since he wants meanings to be \e{intersubjective}.  If meanings were not intersubjective, then successful communication would not be possible. 
	\p So Frege tells us that senses are not ideas, and that they are intersubjective---see p.~39, ``the sign's sense...may be the common property of many...In light of this, one need have no scruples in speaking simply of \e{the} sense''.  (Although see the curious footnote on p.~37.)
	\p See the telescope analogy on pp.~39--40.  The moon is like the referent---it can be viewed through means other than the telescope, and exists independent of the telescope.  The image on our retina is like the idea---yours and mine differ.  The image projected through the interior of the telescope is like the sense---it is a mode of presentation which you and I may both use to view the moon.
	\p As already mentioned, expressions in different languages can share the same sense.
	\p Sense determines reference, but the converse does not hold.  Thus, the sense of `Hesperus' determines that `Hesperus' refers to the planet Venus, but the planet Venus does not determine the sense of `Hesperus'---for there are multiple senses which will determine the referent of the planet Venus (\e{e.g.}, the sense of `Phosphorus').
	\p It is possible for an expression to have sense without reference.  For instance, `Santa Clause' and `the greatest number' both have sense, but neither has reference.
	\ze 
\p Not only do \e{names} like `Hesperus' and `Phosphorus' have senses---\e{all} referring expressions have senses.  What's more, according to Frege, \e{whole sentences} are referring expressions.
	\qe
	\p Why should we take whole sentences to refer at all?  I see the following (implicit, and abductive) argument in Frege: Frege claims that just as with names, whole sentences may have sense without reference.  Thus, `Santa Claus is jolly' and `The greatest number is prime' will have sense without reference.  Thus, by saying that sentences may have referents, we can distinguish between sentences like `Santa Claus is jolly' and `Hesperus is Phosphorus'.  One bit of theoretical work that the referents of sentences does, then, is this: it allows us to distinguish sentences which suffer from \e{reference failure} and those that do not.
	\p If whole sentences refer, then what do they refer to?  Frege says that they refer to their truth-values (`The True' or `The False').    The argument for this (p.~42) isn't entirely satisfying: ``why is [the sense of the sentence] not enough for us?  Because, and to the extent that, we are concerned with its truth-value...We are therefore driven to accepting the \e{truth value} of a sentence as constituting its reference.''
	\p Notice that this has the consequence that all true sentences have the same referent, and all false sentences have the same referent.  An argument for this conclusion is the so-called \e{slingshot} argument.  Given a sentence $s$, let `$r(s)$' denote the referent of $s$.  We may suppose that any two logically equivalent sentences have the same referent.  And we will suppose that, while exchanging a referring expression with another, co-referring expression may make a difference to the \e{sense} of a sentence, it will not make a different to the \e{reference} of the sentence.  Then,  consider any two true sentences, $p$ and $q$.  The following argument establishes that the referent of $p$ is the referent of $q$.
		\argu{200pt}{
			\p[P1.] $r(p) = r(\{ x \mid x=x \wedge p \} = \{ x \mid x=x \})$
			\p[P2.] $\{ x \mid x=x \wedge p \} = \{ x \mid x=x \wedge q \}$
			\vspace{5pt} \hrule
			\p[C1.] $r(p) = r( \{ x \mid x=x \wedge q \} = \{ x \mid x=x \} )$
			\p[P3.] $r(q) = r( \{ x \mid x=x \wedge q \} = \{ x \mid x=x \} )$
			\vspace{5pt}\hrule
			\p[C2.] $r(p) = r(q)$
			}
	P1 and P3 follow from our assumption that logically equivalent sentences have the same referent.  P2 is true because we've assumed that both $p$ and $q$ are true.   C1 follows from P1 and P2 by our assumption that exchanging a referring expression with another, co-referring expression does not make a difference to the reference of the expression.  And C2 follows from C1 and P3 by the symmetry and transitivity of identity.
	\ze 
\p Frege calls the senses of whole sentences \e{thoughts}.
	\qe
	\p (\tbf{A}) and (\tbf{B}) correspond to different thoughts; this difference in meant to explain why (\tbf{A}) is \e{a priori} while (\tbf{B}) is not, how (\tbf{B}) constitutes an advance in our knowledge, while (\tbf{A}) does not.  That is, differences in thought correspond to differences in cognitive significance. \label{a}
	\p For Frege, a thought is the bearer of truth and falsehood.  If the referent of the sentence $s$ is the True, then the thought expressed by $s$ is true.  If the referent of the sentence $s$ is the False, then the thought expressed by $s$ is false. \label{b}
	\p Thoughts are preserved in translation.  Thus, `Hesperus $=$ Phosphorus' and  `$E \sigma \pi \epsilon \rho o \varsigma = \Phi \omega \sigma \varphi$\'{o}$\rho o \varsigma$' express the same thought. \label{c}
	\p For reasons we will soon discuss, thoughts are also taken to be the objects of belief---that is, they are taken to be the referents of \e{that}-clauses in belief reports.  That is, when we say ``John believes that Hesperus $=$ Hesperus'', the referent of ``that Hesperus $=$ Hesperus'' is the \e{thought} expressed by  ``Hesperus $=$ Hesperus'', and not the True. \label{d}
	\ze 
\p Nowadays, the objects which play the theoretical roles \eqref{a}---\eqref{d} are called \e{propositions}.   As we will see, while there is general agreement that there is a single entity which plays the theoretical roles of \eqref{b}, \eqref{c}, and \eqref{d}, whether this entity \e{also} plays the theoretical role of explaining differences in cognitive significance, \eqref{a}, is controversial.  (That is: it's controversial whether the thing that explains differences in cognitive significance is also the primary bearer of truth and falsehood, the object of belief, and the thing preserved in translation).
\p Frege uses the theoretical role \eqref{a} to provide a criterion of how to individuate thoughts.  If a rational person could believe that the sentence $s$ is true while believing that the sentence $s^*$ is false (or \e{vice versa}), then $s$ and $s^*$ express different thoughts.  And, going in the other direction, if no rational person could believe that $s$ is true while believing that $s^*$ is false (or \e{vice versa}), then $s$ and $s^*$ express the same thought.
	\qd
	\p[\s{Individuation of Thoughts}] The sentences $s$ and $s^*$ express different thoughts iff a rational person could believe that $s$ and $s^*$ have different truth-values.
	\zd 
	\qe
	\p Thus, \e{e.g.}, ``Mark Twain wrote Huckleberry Finn''  expresses a different thought than ``Sam Clemens wrote Huckleberry Finn''.  And ``John ate an apple'' and ``An apple was eaten by John'' will express the same thought.
	\ze 
	
\section{Frege's Second Puzzle}
\p The difference between (\tbf{A}) and (\tbf{B}) has to do entirely with the \e{thoughts} they express---that is, it has entirely to do with the \e{senses} of the sentences.  At the level of reference, there is no difference.  All of the component parts have the same reference, and so too does the whole---both sentences refer to the True.
\p This suggests the following hypothesis: compositionality holds, not only for \e{meanings}, but also for \e{reference}.
		\qd
		\p[\s{Compositionality for Reference}]  The referents of whole sentences are determined by the referents of the parts which compose them.
		\zd 
		\qe
		\p \s{Compositionality for Reference} entails the principle that co-referring terms are substitutable \e{salva veritate}.  This principle says that, if `$x$' and `$y$' have the same referent, and if $s[x]$ is a true sentence containing `$x$', then $s[y / x]$---the result of going through $s$ and replacing some occurrence of `$x$' with `$y$'---will also be true.
		\p Thus, if `Jeffrey' and `Dmitri' have the same {referent}, then the sentences `Jeffrey is bald' and `Dmitri is bald' will have the same referent---that is, either both will be true or both will be false (or perhaps both will fail to refer).
		\ze 
\p Suppose that \s{Compositionality for Reference} is true.  Then, co-referring terms will be substitutable \e{salva veritate}.  So the following argument will be valid:
	\argu{260pt}{
	\p[P4.]  `The Greeks believed that Hesperus $=$ Hesperus' is true.
	\p[P5.] `Hesperus' and `Phosphorus' have the same referent.
	\vspace{5pt} \hrule
	\p[C3.]  `The Greeks believed that Hesperus $=$ Phosphorus' is true.
	}
\p This is Frege's second puzzle.  Recall, the first puzzle was that (\tbf{A}) and (\tbf{B}) had different \e{cognitive significance}---for instance, one was \e{a priori}, the other was not, even though `Hesperus' and `Phosphorus' co-refer.  The second puzzle is that (\tbf{F}) and (\tbf{G}) have different \e{truth-value}, even though `Hesperus' and `Phosphorus' co-refer.
		\qe
		\p[(\tbf{F})] The Greeks believed that Hesperus $=$ Hesperus.
		\p[(\tbf{G})] The Greeks believed that Hesperus $=$ Phosphorus.
		\ze 
		\qe
		\p The distinction between sense and reference, on its own, does nothing to solve this puzzle.  That distinction allows us to say that (\tbf{F}) and (\tbf{G}) express different \e{thoughts}.  It does not, on its own, allow us to say that they designate different \e{referents}.
		\ze 
\p Frege's solution to the second puzzle is to deny premise P5 of the argument above.  He denies that, in sentences like (\tbf{G}), `Hesperus' and `Phosphorus' have the same referent.
		\qe
		\p Compare the argument above to the following fallacious argument:
				\argu{240pt}{
				\p[P6.]  Dmitri wrote `Hesperus $=$ Hesperus' on the board.
				\p[P7.] `Hesperus' and `Phosphorus' have the same referent.
				\vspace{5pt} \hrule
				\p[C4.]  Dmitri wrote `Hesperus $=$ Phosphorus' on the board.
				}
		\p This argument is invalid because, in the context of the sentence `Dmitri wrote `Hesperus $=$ Hesperus' on the board', `Hesperus' does \e{not} refer to Venus.  Even though `Hesperus' refers to Venus, ``Hesperus'' refers to the \e{word} `Hesperus'.  Enclosing a word in quotation marks is a way of changing its reference from its usual referent to the word itself.
		\p Similarly, Frege claims that, in P4, `Hesperus' does not refer to Venus.  Rather, `Hesperus' refers to the usual \e{sense} of the word `Hesperus'.
		\p Just as enclosing `Hesperus' in quotation marks changes its reference from the planet Venus to the \e{word} `Hesperus', so too does embedding a sentence in a belief report change the sentence's referent from its truth-value to the \e{thought} the sentence expresses.
		\p That is, in the sentence (\tbf{F}),
				\qe
				\p[(\tbf{F})] The Greeks believed that Hesperus $=$ Hesperus
				\ze 
			`Hesperus $=$ Hesperus' does not refer to the True; but, instead, it refers to the \e{thought} expressed by `Hesperus $=$ Hesperus'.
		\ze 
\p A \e{sentential context} is, roughly, a position within a sentence.  
	\qe
	\p So, for instance, in the sentence
	\begin{center}
	John Wilkes Booth assassinated Lincoln.
	\end{center}
`Lincoln' occurs in a certain sentential position---it fills the blank in 
	\begin{center}
	John Wilkes Booth assassinated \underline{~~~~~~~~~~~~~~~}.
	\end{center}	 
	This position in the sentence is a sentential context.  
		\p Similarly, in the sentence
	\begin{center}
	Dmitri fears the monster at the end of the book.
	\end{center}
`the monster at the end of the book' occurs in a certain sentential context---it fills in the blank in 
	\begin{center}
	Dmitri fears \underline{~~~~~~~~~~~~~~~~~~~~~~~~~~~~~~~~~~~~~~~~~~~~~~~~~~~~~}.
	\end{center}
	\p And, in the sentence 
		\begin{center}
		`Venus' has five letters.
		\end{center}
	`Venus' occurs in the context
		\begin{center}
		`\underline{~~~~~~~~~~}' has five letters.
		\end{center}
	\ze 
\p A sentential context is called \e{transparent} (or \e{extensional}) if, in that context, you can substitute co-referring terms \e{salva veritate}.  And a sentential context is called \e{opaque} (or, perhaps misleadingly, \e{intensional}) if, in that context, substituting co-referring terms can change the truth-value of the expression.
	\qe
	\p In `John Wilkes Booth assassinated Lincoln', `Lincoln' uncontroversially appears in a transparent context.  If `John Wilkes Booth assassinated Lincoln' is true, and `Lincoln'  and `the 16th President of the United States' co-refer, then `John Wilkes Booth assassinated the 16th President of the United States' is true also.
	\p In ``Venus' has five letters', `Venus' uncontroversially appears in an opaque context.   ``Venus' has five letters' is true, `Venus' and `Hesperus' co-refer, yet ``Hesperus' has five letters' is false.
	\p Frege's claim is that, when it comes to sentences like `Dmitri fears the monster at the end of the book' (sentences involving \e{attitude verbs} like \e{fear}, \e{believe}, \e{know}, \e{hope}, and so on), the context following the attitude verb (known as the \e{prejacent} of the attitude verb) is opaque. 
		\qe
		\p Thus, even though `the monster at the end of the book' and `Grover' co-refer, and `Dmitri fears the monster at the end of the book' is true, `Dmitri fears Grover' is false.
		\p Similarly, even though `Hesperus' and `Phosphorus' co-refer, and `The Greeks believed Hesperus $=$ Hesperus' is true, `The Greeks believed Hesperus $=$ Phosphorus' can still be false.
		\ze  
	\ze 
	\newpage
\p An argument that attitude verbs do not create opaque contexts (applied to the case of `Hesperus' and `Phosphorus'):
		\argu{330pt}{
		\p[P8.] The Greeks knew that Hesperus was the evening star.
		\vspace{5pt}\hrule
		\p[C5.] Hesperus has the property of being known by the Greeks to be the evening star.
		\p[P9.] Hesperus $=$ Phosphorus.
		\p[P10.] For all $a, b$, if $Fa$ and $a = b$, then $Fb$.
		\vspace{6pt}\hrule
		\p[C6.] Phosphorus has the property of being known by the Greeks to be the evening star.
		\vspace{0pt}\hrule
		\p[C7.] The Greeks knew that Phosphorus was the evening star.
		}
\ze 



\end{document}