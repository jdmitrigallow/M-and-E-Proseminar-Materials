\documentclass[12pt]{beamer}
\usepackage{tikz}
\usetikzlibrary{arrows}
\usepackage{kbordermatrix}
\usepackage[adobe-garamond]{mathdesign}
\usepackage{fontspec}
\usetheme[titleformat=regular,sectionpage=progressbar, subsectionpage=progressbar, background=light]{metropolis}
\setsansfont[Ligatures=TeX,Numbers={Proportional,OldStyle}]{Minion Pro}
\setmonofont[Ligatures=TeX,Numbers={Proportional,OldStyle}]{Minion Pro}
\AtBeginDocument{%
  \let\mathbb\relax
  \DeclareMathAlphabet\PazoBB{U}{fplmbb}{m}{n}%
  \newcommand{\mathbb}{\PazoBB}%
  \let\mathcal\relax
  \DeclareMathAlphabet{\OMScal}{OMS}{cmsy}{m}{n}
  \newcommand{\mathcal}{\OMScal}%
}
\usepackage{appendixnumberbeamer}
%GRAPHICX PACKAGE
\usepackage{graphicx}
\graphicspath{{/Users/jdg83/Dropbox/0000Desktop/Figures/}}

\usepackage[normalem]{ulem}
\usepackage{xcolor}
\newcommand\redsout{\bgroup\markoverwith{\textcolor{red}{\rule[0.5ex]{2pt}{1pt}}}\ULon}

\usepackage{booktabs}
\usepackage[scale=2]{ccicons}

\usepackage{pgfplots}
\usepgfplotslibrary{dateplot}

%\usepackage{natbib}

\newcommand{\argu}[2]{\begin{center}\begin{minipage}{#1} \begin{enumerate}[<+->]
	#2
\end{enumerate}
\end{minipage}  
\end{center}}
\newcommand{\thus}{

\vspace{10pt}

\hrule

\vspace{5pt}

}
\newcommand{\qq}[1]{~\ulcorner #1  \urcorner~}
\newcommand{\V}[1]{\llbracket #1 \rrbracket}
\newcommand{\D}{\mathcal{D}}
\newcommand{\W}{\mathcal{W}}
\newcommand{\K}{\mathcal{K}}
\renewcommand{\u}{\mathfrak{u}}
\newcommand{\df}{\stackrel{\text{\tiny def}}{=}}
\newcommand{\s}{\textsc}
\newcommand{\e}{\emph}
\newcommand{\tbf}{\textbf}
\definecolor{dblue}{HTML}{0913c6}
\definecolor{dred}{HTML}{990707}
\renewcommand{\b}[1]{\textcolor{dblue}{#1}}
\renewcommand{\r}[1]{\textcolor{dred}{#1}}
\newcommand{\g}[1]{\textcolor{gray!75}{#1}}
\renewcommand{\o}[1]{\textcolor{orange}{#1}}
\newcommand\Fontbig{\fontsize{15}{20}\selectfont}
%%%%%%%
\newcommand{\qe}{\begin{itemize}[<+->]}
\newcommand{\p}{\item}
\newcommand{\ze}{\end{itemize}}

\newcommand{\qf}[2]{\begin{frame}{#1}
#2
\end{frame}
}

\DeclareMathOperator*{\argmax}{\text{arg\,max}}

\title{\emph{Skepticism {\&} Contextualism}}
\date{November 6th, 2018}
\author{}
\institute{}
% \titlegraphic{\hfill\includegraphics[height=1.5cm]{pitt}}

\begin{document}

\maketitle

%\begin{frame}[standout]
%  Please interrupt when I stop making sense.
%\end{frame}

\section{A Modal Logic for Knowledge}

\qf{Symbolization}{
\qe
\p Represent
	\begin{center}
	$S$ is in a position to know that $\phi$
	\end{center}
with
	\[
	\alt<3->{\K \phi}{\K_{\alert<2>{S}} \phi }
	\]
\ze 
}

\qf{`In a position to know'}{
	\begin{alertblock}
	{Closure}
	For any propositions $\phi$ and $\psi$,
		\[
		[\K\phi \wedge \alt<3, 4>{\alert<3>{\K(\phi \text{ entails } \psi) }}{\alert<2>{\phi \text{ entails } \psi}}] \to \K \psi
		\]
	\end{alertblock}
}

\qf{Kripke Models}{
	\qe
	\p A \e{Kripke model} $< \W, R, V>$ contains:
		\qe
		\p A set $\W$ of possible worlds.
		\p A binary relation $R \subset \W \times \W$.
		\p A function $V$ from atomic propositions to subsets of $\W$.
		\ze 
	\ze 
}

\qf{Kripke Models}{
\begin{center}
{\LARGE
\begin{tikzpicture}[scale=3.5,>=stealth']
\tikzset{
    p1/.style={%
        draw=blue, thick,
        rectangle,
        rounded corners=10pt,
        minimum height=160pt,
        minimum width=60pt
        },
    q1/.style={%
        draw=red, thick,
        rectangle,
        rounded corners=10pt,
        minimum height=70pt,
        minimum width=160pt
        },
}
%%%PROPOSITIONS
\node[p1, fill=blue, fill opacity=0.1] (c2) at (0,0.5) {};
\node[q1, fill=red, fill opacity=0.1] (c1)  at (0.5,-0.05) {};
\filldraw[red]  (1.6, -0.05) node[anchor=east] { $q$ };
\filldraw[blue]  (-0.55, 1) node[anchor=west] { $p$ };
%%%WORLDS
\filldraw [black] 
(0,0) circle (0.5pt) node[anchor=north, below=4pt] { $w_1$ } 
(0, 1) circle (0.5pt)   node[anchor=south, above=4pt] { $w_2$ } 
(1,1) circle (0.5pt) node[anchor=south, above=4pt] { $w_3$ } 
(1,0) circle (0.5pt)   node[anchor=north, below=4pt] { $w_4$ } 
;
%%%ACCESSIBILITY RELATIONS 
\draw[->,thick] (0, 0.1) -- (0, 0.9);
\draw[<-, thick] (0.1,1) -- (0.9, 1);
\draw[->, thick] (0.08, 0.08) -- (0.92, 0.92);
\draw[<-, thick] (-0.05,0.05) .. controls (-0.3, 0.2) and (-0.3, -0.2) .. (-0.05, -0.05);
\draw[<-, thick] (1.05,0.05) .. controls (1.3, 0.2) and (1.3, -0.2) .. (1.05, -0.05);
\end{tikzpicture}
}
\end{center}
}

\qf{Kripke Models}{
	\begin{center}
	{\LARGE
	\begin{tikzpicture}[scale=3.5,>=stealth']
\tikzset{
    p1/.style={%
        draw=blue, thick,
        rectangle,
        rounded corners=10pt,
        minimum height=80pt,
        minimum width=70pt
        },
    q1/.style={%
        draw=red, thick,
        rectangle,
        rounded corners=10pt,
        minimum height=80pt,
        minimum width=150pt
        },
}
%%%PROPOSITIONS
\node[p1, fill=blue, fill opacity=0.1] (c2) at (-1,0.05) {};
\node[q1, fill=red, fill opacity=0.1] (c1)  at (0.5,0.05) {};
\filldraw[red]  (1.6, 0) node[anchor=east] { $q$ };
\filldraw[blue]  (-1.6, 0) node[anchor=west] { $p$ };
%%%WORLDS
\filldraw [black] 
(-1,0) circle (0.5pt) node[anchor=south, above=4pt] { $w_1$ } 
(0, 0) circle (0.5pt)   node[anchor=south, above=4pt] { $w_2$ } 
(1,0) circle (0.5pt) node[anchor=south, above=4pt] { $w_3$ } 
;
%%%ACCESSIBILITY RELATIONS 
\draw[->,thick] ( -0.1, 0) -- (-0.9,0);
%\draw[<-, thick] (0.1,1) -- (0.9, 1);
%\draw[->, thick] (0.08, 0.08) -- (0.92, 0.92);
\draw[<-, thick] (-1.05,-0.05) .. controls (-1.3, -0.3) and (-0.7, -0.3) .. (-0.95, -0.05);
\draw[->, thick] (1.05,-0.05) .. controls (1.3, -0.3) and (0.7, -0.3) .. (0.95, -0.05);
\end{tikzpicture}
	}
	\end{center}
}

\qf{Kripke Models}{
	\begin{center}
	{\LARGE
	\begin{tikzpicture}[scale=3.5,>=stealth']
\tikzset{
    p1/.style={%
        draw=blue, thick,
        rectangle,
        rounded corners=10pt,
        minimum height=70pt,
        minimum width=150pt
        },
    q1/.style={%
        draw=red, thick,
        rectangle,
        rounded corners=10pt,
        minimum height=70pt,
        minimum width=150pt
        },
}
%%%PROPOSITIONS
\node[p1, fill=blue, fill opacity=0.1] (c2) at (-0.5,0.05) {};
\node[q1, fill=red, fill opacity=0.1] (c1)  at (0.5,0.05) {};
\filldraw[red]  (1.6, 0) node[anchor=east] { $q$ };
\filldraw[blue]  (-1.6, 0) node[anchor=west] { $p$ };
%%%WORLDS
\filldraw [black] 
(-1,0) circle (0.5pt) node[anchor=south, above=4pt] { $w_1$ } 
(0, 0) circle (0.5pt)   node[anchor=south, above=4pt] { $w_2$ } 
(1,0) circle (0.5pt) node[anchor=south, above=4pt] { $w_3$ } 
;
%%%ACCESSIBILITY RELATIONS 
\draw[->,thick] ( -0.1, 0) -- (-0.9,0);
%\draw[<-, thick] (0.1,1) -- (0.9, 1);
%\draw[->, thick] (0.08, 0.08) -- (0.92, 0.92);
\draw[<-, thick] (-1.05,-0.05) .. controls (-1.3, -0.3) and (-0.7, -0.3) .. (-0.95, -0.05);
\draw[->, thick] (1.05,-0.05) .. controls (1.3, -0.3) and (0.7, -0.3) .. (0.95, -0.05);
\draw[->, thick] (.05,-0.05) .. controls (0.3, -0.3) and (-0.3, -0.3) .. (-0.05, -0.05);
\end{tikzpicture}
	}
	\end{center}
}


\qf{Kripke Models}{
	\begin{center}
	{\LARGE
	\begin{tikzpicture}[scale=3.5,>=stealth']
\tikzset{
    p1/.style={%
        draw=blue, thick,
        rectangle,
        rounded corners=10pt,
        minimum height=80pt,
        minimum width=250pt
        },
    q1/.style={%
        draw=red, thick,
        rectangle,
        rounded corners=10pt,
        minimum height=80pt,
        minimum width=90pt
        },
}
%%%PROPOSITIONS
\node[p1, fill=blue, fill opacity=0.1] (c2) at (0,0.02) {};
\node[q1, fill=red, fill opacity=0.1] (c1)  at (0,-1) {};
\filldraw[red]  (-0.45, -1) node[anchor=east] { $q$ };
\filldraw[blue]  (-1.6, 0) node[anchor=west] { $p$ };
%%%WORLDS
\filldraw [black] 
(-1,0) circle (0.5pt) node[anchor=south, above=4pt] { $w_1$ } 
(0, -1) circle (0.5pt)   node[anchor=west, right=4pt] { $w_2$ } 
(1,0) circle (0.5pt) node[anchor=south, above=4pt] { $w_3$ } 
(2,0) circle (0.5pt) node[anchor=south, above=4pt] { $w_4$ } 
;
%%%ACCESSIBILITY RELATIONS 
\draw[<->,thick] ( -0.05, -0.95) -- (-0.9,-0.04);
\draw[<->, thick] (0.05,-0.95) -- (0.87, -0.04);
\draw[<->, thick] (-0.9,0) -- (0.87, 0);
%\draw[->, thick] (0.08, 0.08) -- (0.92, 0.92);
\draw[<-, thick] (-1.05,-0.05) .. controls (-1.3, -0.3) and (-0.7, -0.3) .. (-0.95, -0.05);
\draw[<-, thick] (1.05,-0.05) .. controls (1.3, -0.3) and (0.7, -0.3) .. (0.95, -0.05);
\draw[->, thick] (.05,-1.05) .. controls (0.3, -1.3) and (-0.3, -1.3) .. (-0.05, -1.05);
\draw[->, thick] (2.05,-0.05) .. controls (2.3, -0.3) and (1.7, -0.3) .. (1.95, -0.05);
\end{tikzpicture}
	}
	\end{center}
}

\qf{Semantics for $\K$}{
	\qe
	\p We then define an \e{interpretation function}, $\V{\phi}^w$, from sentences, $\phi$, and worlds, $w$, to truth-value, $\{ 0, 1 \}$
	\p For any atomic sentence $\alpha$, $\V{\alpha}^w = 1$ iff $w \in V(\alpha)$.
	\p For any sentence $\phi$, $\V{\neg \phi }^w = 1$ iff $\V{\phi}^w = 0$.
	\p For any sentences $\phi$, $\psi$, $\V{ \phi \to \psi }^w  = 1$ iff $\V{\phi}^w = 0$ or $\V{\psi}^w = 1$.
	\ze 
}

\qf{Semantics for $\K$}{
	\qe
	\p For any sentence $\phi$, 
	
	\p<1->[]	$\V{\K \phi}^{\alert<2>{w}} = 1$ iff $\V{\phi}^{\alert<2>{w^*}} = 1$ for all $w^*$ such that $Rww^*$. 
	\ze 
}

\qf{Propositions}{
	\qe
	\p For any sentence $\phi$,
				\[
				\langle \phi \rangle \df \{ w \in \W \mid \V{\phi}^w = 1 \}
				\]
	\ze 
}

\qf{Semantics for $\K$}{
	\qe
	\p A Kripke model: $< \W, R, V>$.
	\p We should think of the worlds $\W$ as \e{centered} possibilities to allow for knowledge \e{de se}
	\p We should think of $R$ like this:
		\qe
		\p $R w w^*$ iff $w^*$ is consistent with $S$'s evidence at $w$.
		\ze 
	\p So, our semantics says:
		\qe
		\p \alt<7->{$S$ knows that $\phi$ iff $\phi$ is entailed by $S$'s evidence.}{$S$ knows that $\phi$ iff $\neg \phi$ is inconsistent with $S$'s evidence}
		\ze 
	\ze 
}

\qf{Semantics for $\K$}{
	\qe
	\p Let $E_w \df \{ w^* \in \W \mid  Rww^* \}$.
	\p Then, for any sentence $\phi$,
			\[
			\V{\K \phi}^w = 1 \quad \text{ iff } \quad E_w \subseteq \langle \phi \rangle 
			\]
	\ze 
}

\qf{Semantics for $\K$}{
	\begin{center}
\alt<2->{
\includegraphics[scale=0.27]{know.eps}
}{	\includegraphics[scale=0.27]{noknow.eps}
}
	\end{center}
}

\qf{Semantics for $\K$}{
	\begin{alertblock}
	{Reflexivity}
	
	For any $w$, $Rww$ 
	
	 [For any $w$, $w \in E_w$]
	\end{alertblock}

\uncover<2->{	\begin{alertblock}
	{Symmetry}
	
	For any $w$, $w^*$,  if $Rww^*$, then $Rw^*w$. 
	
	 [For any $w$, $w^*$, if $w^* \in E_w$, then $w \in E_{w^*}$]
	\end{alertblock}}
	
\uncover<3->{	\begin{alertblock}
	{Transitivity}
	
	For any $w, w^*, w^{**}$, if $Rww^*$ and $Rw^* w^{**}$, then $Rww^{**}$.
	
	[For any $w, w^*, w^{**}$, if $w^* \in E_w$ and $w^{**} \in E_{w^*}$, then $w^{**} \in E_w$]
	\end{alertblock}}
}

\qf{Semantics for $\K$}{
	\begin{alertblock}
	{Factivity} 
	$\K \phi \to \phi$   \hfill [(T)]
	\end{alertblock}
	
\uncover<2->{	\begin{alertblock}
	{Brouwer}
	
	$\neg \K \neg \K \phi \to \phi$ \hfill [(B)]
	\end{alertblock}}
	
\uncover<3->{	\begin{alertblock}
	{Positive Introspection}
	
$\K \phi \to \K \K \phi$ \hfill [(S4)]
\end{alertblock}	 }

\uncover<4->{
	\begin{alertblock}
	{Negative Introspection}
	
	$\neg \K \phi \to \K \neg \K \phi$ \hfill [(S5)]
	\end{alertblock}
}
}

\section{A Skeptical Argument}

\qf{A Skeptical Argument}{
		\argu{250pt}{
	\p[P1.] $S$ knows that they have hands only if they know that they are not a handless brain in a vat.
	\p[P2.] $S$ does not know that they are not a handless brain in a vat.
	\thus
	\p[C1.] $S$ does not know that they have hands.
	}
}

\qf{A Skeptical Argument}{
		\argu{90pt}{
		\p[P1.] \alert<4>{$\K p \to \K \neg s$}
		\p[P2.] \alert<5>{$\neg \K \neg s$}
		\thus
		\p[C1.] \alert<6>{$\neg \K p$}
		} 
		\qe
		\p<4-> \alert<4>{Deny P1}
		\p<5-> \alert<5>{Deny P2}
		\p<6-> \alert<6>{Accept C1}
		\ze 
}

\qf{Accept C1?  \hfill  $\neg \K p$}{
	\qe
	\p Knowledge plays important normative roles.
		\qe
		\p Excuses
		\p Assertion
		\ze 
	\ze 
}

\qf{Deny P2? \hfill $\K \neg s$}{
\argu{290pt}{
	\p<2->[P3.] A handless brain-in-a-vat, being stimulated to have experiences indistinguishable from the experiences of $S$, doesn't know that they have hands.
	\p<3->[P4.] A handless brain-in-a-vat, being stimulated to have experiences indistinguishable from the experiences of $S$, has exactly the same evidence that $S$ has.
	\p<4->[P5.] If two people have exactly the same evidence, then either both  know that $\phi$ or neither does.
	\thus
	\p<5->[C2.] $S$  doesn't know that they have hands.
	}
}

\qf{Deny P2? \hfill $\K \neg s$}{
\argu{290pt}{
	\p<2->[P6.] At $w_s$, $S$'s evidence does not rule out $w$.
	\p<3->[P7.] \tbf{Symmetry}
	\thus
	\p<4->[C3.] At $w$, $S$'s evidence does not rule out $w_s$.
	\p<5->[P8.] $S$ knows that $\phi$ only if $S$'s evidence rules out all $\neg \phi$ possibilities.
	\thus
	\p<6->[C4.] $S$ doesn't know that they're not a handless brain-in-a-vat.
	}
}

\qf{Deny P1? \hfill $\K p \wedge \neg \K \neg s$}{
\uncover<2->{	\begin{alertblock}
	{Closure}
	For any propositions $\phi$ and $\psi$,
		\[
		[\K\phi \wedge \K(\phi \text{ entails } \psi)] \to \K \psi
		\]
	\end{alertblock}}
}

\section{A Contextualist Modal Logic for Knowledge}

\qf{Kripke Models}{
	\qe
	\p A \e{Kripke model} $< \W, R, V>$ contains:
		\qe
		\p A set $\W$ of possible worlds.
		\p A binary relation $R \subset \W \times \W$.
		\p A function $V$ from atomic propositions to subsets of $\W$.
		\ze 
	\ze 
}

\qf{Contextualist Semantics for $\K$}{
	\qe
	\p We then define a \e{contextualized} interpretation function, $\V{}$, which is a function from a sentence $\phi$, a \e{context of utterance} $\mathcal{C}$, and a possible world $w$, to truth-value, $\{ 1, 0 \}$.
\p $\mathcal{C}$ is a \e{set of contextually relevant worlds}.
	\ze 
}

\qf{Contextualist Semantics for $\K$}{
	\qe
	\p For any atomic sentence $\alpha$, $\V{\alpha}^{\mathcal{C}, w} = 1$ iff $w \in V(\alpha)$.
	\p For any sentence $\phi$, $\V{\neg \phi }^{\mathcal{C}, w} = 1$ iff $\V{\phi}^{\mathcal{C}, w} = 0$.
	\p For any sentences $\phi$, $\psi$, $\V{ \phi \to \psi }^{\mathcal{C}, w}  = 1$ iff $\V{\phi}^{\mathcal{C}, w} = 0$ or $\V{\psi}^{\mathcal{C}, w} = 1$.
	\ze 
}

\qf{Contextualist Semantics for $\K$}{
	\qe
	\p For any sentence $\phi$, 
	
	\p<1->[]	$\V{\K \phi}^{\mathcal{C}, w} = 1$ \quad iff \quad $E_w \alert<2>{\cap \mathcal{C} }\subseteq \langle \phi \rangle^{\mathcal{C}}$.
	\ze 
}

\qf{Contextualist Semantics for $\K$}{
	\begin{center}
\alt<2->{
\includegraphics[scale=0.27]{noknowC.eps}
}{	\includegraphics[scale=0.27]{knowC.eps}
}
	\end{center}
}

\section{Lewis's Contextualism}

\qf{Lewis's Contextualism}{
	\qe
	\p $E_w$ is the set of worlds in which $S$ has the same experience they actually have.
	\p Lewis characterizes $\mathcal{C}$ by way of a series of \e{rules} about which possibilities may and may not be properly ignored.
	\ze 
}

\qf{Rules of Prohibition}{
\uncover<2->{	\begin{alertblock}
	{Rule of Actuality}
	
	Do not ignore $S$'s actual world---that world must be included in $\mathcal{C}$.
	\end{alertblock} }
	
\uncover<3->{		\begin{alertblock}
	{Rule of Belief}
	
Do not ignore worlds that $S$ believes (or ought to believe) to obtain---those worlds must be included in $\mathcal{C}$.
	\end{alertblock}}
	
\uncover<4->{		\begin{alertblock}
	{Rule of Resemblance}
	
Do not ignore worlds that saliently resemble worlds you have already not ignored---those worlds must be included in $\mathcal{C}$.
	\end{alertblock}}
}

\qf{Rules of Permission}{
\uncover<2->{	\begin{alertblock}
	{Rule of Reliability}
	
So long as doing so doesn't violate the preceeding rules, possibilities in which reliable processes like perception, memory, and testimony fail may be properly ignored---they may be excluded from $\mathcal{C}$.
	\end{alertblock} }
	
\uncover<3->{		\begin{alertblock}
	{Rule of Method}
	
So long as doing so doesn't violate the preceeding rules, possibilities in which inductive inference or inference to the best explanation fail may be properly ignored---they may be excluded from $\mathcal{C}$.
	\end{alertblock}}

}

\qf{Rules of Permission}{
\begin{alertblock}
	{Rule of Conservativism}
	
If it is common knowledge in our linguistic community that we standardly \e{do} ignore possibilities, then these possibilities may be ignored---they may be excluded from $\mathcal{C}$.
	\end{alertblock}
}

\qf{Rule of Attention}{
\begin{alertblock}
	{Rule of Attention}
A possibility which is not being ignored is not being properly ignored.
	\end{alertblock}
}


\end{document}
